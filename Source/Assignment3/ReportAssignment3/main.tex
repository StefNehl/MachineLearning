%%%%%%%%%%%%%%%%%%%%%%%%%%%%%%%%%%%%%%%%%
% Chair of Cyber-Physical-Systems
% Univ.-Prof. Dr. Elmar Rueckert
% Montanuniversität Leoben, Austria
% Latest Update: Feb. 2022
%
% Disclaimer: The materials and source code are for personal use only. The material is intended for educational purposes only. Reproduction of the material for any purposes other than what is intended is prohibited. The content is to be used for educational and non-commercial purposes only and is not to be changed, altered, or used for any commercial endeavor without the express written permission of Professor Rueckert. 
% 
% Original Version by Frits Wenneker, 28/2/17,  License: CC BY-NC-SA 3.0 (http://creativecommons.org/licenses/by-nc-sa/3.0/)
%%%%%%%%%%%%%%%%%%%%%%%%%%%%%%%%%%%%%%%%%

%----------------------------------------------------------------------------------------
%	PACKAGES AND OTHER DOCUMENT CONFIGURATIONS
%----------------------------------------------------------------------------------------

\documentclass[10pt, a4paper, twocolumn]{article} % 10pt font size (11 and 12 also possible), A4 paper (letterpaper for US letter) and two column layout (remove for one column)

\input{structure.tex} % Specifies the document structure and loads requires packages

%----------------------------------------------------------------------------------------
%	ARTICLE INFORMATION
%----------------------------------------------------------------------------------------

\title{Assignment III: Gauss and Beta Distribution} % The article title

\author{
	\coursetitle{Exercises in Machine Learning (190.013), SS2022}
	\authorstyle{Stefan Nehl\textsuperscript{1}} % Authors
	\newline\newline % Space before institutions
	\textsuperscript{1}\textit{stefan-christopher.nehl@stud.unileoben.ac.at, MNr: 00935188}, \institution{Montanuniversität Leoben, Austria}\\ % Institution 1
	\newline\submissiondate{\today} % Add a date here
}

% Example of a one line author/institution relationship
%\author{\newauthor{John Marston} \newinstitution{Universidad Nacional Autónoma de México, Mexico City, Mexico}}


%----------------------------------------------------------------------------------------

\begin{document}
\input{python_code.tex} % To print Python code

\maketitle % Print the title

\thispagestyle{firstpage} % Apply the page style for the first page (no headers and footers)

%----------------------------------------------------------------------------------------
%	ABSTRACT
%----------------------------------------------------------------------------------------

\lettrineabstract{In the third assignment, I had to create the abstract class ContinuousDistribution. This class contains functions for importing and exporting the data to CSV files, compute the mean and standard deviation, generating samples and visualize the given data. Furthermore, I had to implement two additional classes, GaussDistribution and BetaDistribution which inheritances the class ContinuousDistribution and implement gauss and beta distribution. Finally, I plotted the result of this distributions.}

%----------------------------------------------------------------------------------------
%	REPORT CONTENTS
%----------------------------------------------------------------------------------------

\section{Introduction}

The gauss and beta distributions are often used in machine learning to generate random variables. The target of this assignment was to create a module \textit{inference} which implements this two different distributions. 

\section{Implementation}
For the implementation I created a module with the name \textit{inference}. In this module i implemented the abstract class \textit{ContinuousDistribution} with it's abstract methods. 

\section{Abstract class \textit{ContinuousDistribution}}
First the class was \textit{ContinuousDistribution} created. This class is not completely abstract because some methods I reused in the other classes. For this I implemented first a constructor with the default properties. These properties are \textit{dataSet, normalizedDataSet, mean, median, variance} and \textit{standardDeviation}. This properties are reused in the \textit{GaussDistribution} and \textit{BetaDistribution}. The functions, \textit{importCSV, exportCSV, calculateMean, calculateVariance, calculateStandardDeviation} and \textit{normalizeDataSet}. Only the method \textit{generateSamples} and \textit{plotData} is abstract, because every class needs there own implementation of this functions. 

\section{Gauss Distribution}
The class \textit{GaussDistribution} has a constructor with the parameters \textit{dimension}, sets the dimension of this gauss distribution and the optional parameters \textit{fileName}, for importing a \textit{CSV} file, \textit{numberOfSamplesToGenerate}, number of samples generated by the class, \textit{mean} and  \textit{variance}. Important to mention here is, that the importing of a file and the generating of samples is excluding each other. Only one parameter can be set otherwise the class throws an exception. The construction also sets the data and calculate the needed values like mean and standard deviation and generates the gauss distribution. 

\subsection{Generating Samples}
As already mentioned, the generating of samples needs to be implemented for each class separately. For the generation I used the \textit{random()} function from the \textit{numpy} library with the values of the mean and the variance for the generation and the dimension with the number of samples for the amount of data. 

\lstinputlisting[linerange=35-41]{../../Modules/GaussDistribution.py}

\subsection{Calculation}
For the calculation I implemented two different methods. One for the one dimensional calculation, \textit{generateGaussen1D}, and for the two dimensional calculation, \textit{calculateGaussen2D}. For the one dimensional implementation I used the following formula. 
\[
N(x|\mu,\sigma) = \frac{1}{(2\pi\sigma\textsuperscript{2})\textsuperscript{1/2}}\:exp
\textsuperscript{$-\frac{1}{2\sigma\textsuperscript{2}}(x-\mu)\textsuperscript{2}$}
\]
Where $\mu$ stands for the mean and $\sigma$ for the standard deviation. For the two dimensional implementation I used the following formula. 
\[
N(x|\mu,\sigma) = \frac{1}{(2\pi)\textsuperscript{D/2} \vert\Sigma\vert\textsuperscript{1/2}}\:exp\textsuperscript{$-\frac{1}{2}(x-\mu)\textsuperscript{T}\Sigma\textsuperscript{-1}(x-\mu)$}
\]
Where D is the dimension, $\Sigma$ the covariance and $\vert\Sigma\vert$ the determinant of the covariance.For the covariance I used created a vector with the mean and zeros.
$\begin{pmatrix}
\sigma & 0\\
0 & \sigma
\end{pmatrix}$
Both formulas are from the book An Introduction to Probabilistic Machine Learning.  
\citep{bookMachineLearning}

\subsection{Plotting}
The implementation for the one dimensional plot was straightforward. I plotted a histogram of the generated data and the gauss distribution as a line above the histogram. Additional, I add the raw the of the generated samples. With the two dimensional plots I had some issues. I was able to create a 3d model of the raw data and the 3d bar chart of the distribution. I wanted to plot the surface of the two dimensional gauss distribution following the paper \citep{multiVariateNormalDistribution}, but unfortunately I failed to create the surface. 

\section{Beta Distribution}

\subsection{Generating Samples}

\subsection{Calculation}

\subsection{Plotting}

\section{Results}

\section{Conclusion}

\section*{APPENDIX}

\onecolumn\lstinputlisting[]{../../Modules/inference.py}
\onecolumn\lstinputlisting[]{../../Modules/GaussDistribution.py}
\onecolumn\lstinputlisting[]{../../Modules/BetaDistribution.py}



%----------------------------------------------------------------------------------------
%	BIBLIOGRAPHY
%----------------------------------------------------------------------------------------

\printbibliography[title={Bibliography}] % Print the bibliography, section title in curly brackets

%----------------------------------------------------------------------------------------

\end{document}
