%%%%%%%%%%%%%%%%%%%%%%%%%%%%%%%%%%%%%%%%%
% Chair of Cyber-Physical-Systems
% Univ.-Prof. Dr. Elmar Rueckert
% Montanuniversität Leoben, Austria
% Latest Update: Feb. 2022
%
% Disclaimer: The materials and source code are for personal use only. The material is intended for educational purposes only. Reproduction of the material for any purposes other than what is intended is prohibited. The content is to be used for educational and non-commercial purposes only and is not to be changed, altered, or used for any commercial endeavor without the express written permission of Professor Rueckert. 
% 
% Original Version by Frits Wenneker, 28/2/17,  License: CC BY-NC-SA 3.0 (http://creativecommons.org/licenses/by-nc-sa/3.0/)
%%%%%%%%%%%%%%%%%%%%%%%%%%%%%%%%%%%%%%%%%

%----------------------------------------------------------------------------------------
%	PACKAGES AND OTHER DOCUMENT CONFIGURATIONS
%----------------------------------------------------------------------------------------

\documentclass[10pt, a4paper, twocolumn]{article} % 10pt font size (11 and 12 also possible), A4 paper (letterpaper for US letter) and two column layout (remove for one column)

\input{structure.tex} % Specifies the document structure and loads requires packages

%----------------------------------------------------------------------------------------
%	ARTICLE INFORMATION
%----------------------------------------------------------------------------------------

\title{Assignment II: Python Basics (Datafile Import,
Plotting, Functions, Classes)} % The article title

\author{
	\coursetitle{Exercises in Machine Learning (190.013), SS2022}
	\authorstyle{Stefan Nehl\textsuperscript{1}} % Authors
	\newline\newline % Space before institutions
	\textsuperscript{1}\textit{stefan-christopher.nehl@stud.unileoben.ac.at, MNr: 00935188}, \institution{Montanuniversität Leoben, Austria}\\ % Institution 1
	\newline\submissiondate{\today} % Add a date here
}

% Example of a one line author/institution relationship
%\author{\newauthor{John Marston} \newinstitution{Universidad Nacional Autónoma de México, Mexico City, Mexico}}


%----------------------------------------------------------------------------------------

\begin{document}
\input{python_code.tex} % To print Python code

\maketitle % Print the title

\thispagestyle{firstpage} % Apply the page style for the first page (no headers and footers)

%----------------------------------------------------------------------------------------
%	ABSTRACT
%----------------------------------------------------------------------------------------

\lettrineabstract{In the second assignment, I had to create basic statistic functions for calculating the mean, median, variance and standard deviation. Furthermore, I had to read in a CSV file, use the created functions and plot the values.}

%----------------------------------------------------------------------------------------
%	REPORT CONTENTS
%----------------------------------------------------------------------------------------

\section{Introduction}

The statistic functions I implemented were calculating the mean, median, variance, and standard deviation. Furthermore, I added a function for normalizing and standardizing a data set.  

\section{Implementation}
For the implementation I created the class \textit{BasicStatistics} with all the statistic functions. First, I implemented the mean. 

\subsection{Mean}
The mean is the average of a collection of numbers. The formula for the calculation is the following
\citep{meanCFI}: 
\[
mean = \frac{x\textsubscript{1} + x\textsubscript{2} + \ldots + x\textsubscript{n}} {n}
\]
Where n is the number of elements in the collection and x\textsubscript{n} the element on the position n. The calculation was implemented in the \textit{getMean} function. 
\lstinputlisting[linerange=27-30]{../BasicStatistics.py}

\subsection{Median}
The Median is the middle value of a collection of numbers. The steps of the implemented algorithm are
\citep{medianCFI}: 
\begin{itemize}
	\item Sort the collection of numbers 
	\item Calculate the mid index 
	\item if the length of the collection is uneven: take the element with the mid index
	\item if the length is even calculate the median: 
	\[
	median = \frac{x\textsubscript{mid} + x\textsubscript{mid-1}} {2}
	\]
	\item return the result
\end{itemize}


\lstinputlisting[linerange=32-44]{../BasicStatistics.py}

\subsection{Variance}
The variance is the expected variation between values in a collection of numbers. 
The formula for the calculation of the variance it the following
\citep{varianceAndStandardDeviationStack}: 
\[
Variance\:\sigma\textsuperscript{2} = \frac{\sum_{i=0}^{n-1}(x\textsubscript{i} - \overline{x})}{n-1}
\]
Where \textit{n} is the number of elements, \textit{x\textsubscript{i}} the element on the index \textit{i} and \textit{$\overline{x}$} the mean. This formula uses the Bessel's correction for smaller numbers. Therefore, instead of dividing the aggregated values with n, I divided them with n-1 \citep{varianceAndStandardDeviationStack}. 

\lstinputlisting[linerange=46-54]{../BasicStatistics.py}

\subsection{Standard Deviation}
The standard deviation is the amount of the variation of a collection of numbers. 
The formula for the calculation of the standard deviation is
\citep{varianceAndStandardDeviationStack}: 
\[
Standard\:Deviation\:\sigma = \sqrt{\sigma\textsuperscript{2}}
\]

\lstinputlisting[linerange=56-59]{../BasicStatistics.py}


\subsection{Normalize Data}
Standardization is a preprocessing step, to standardize the range of values of a collection of numbers. I used the following formula to standardize the data
\citep{standardizeDataBuildIn}. 
\[
z = \frac{x\textsubscript{i} - \overline{x}}{\sigma}
\]
Where \textit{z} is the standardize value, \textit{$x\textsubscript{i}$} the value in the collection on index i, \textit{$\overline{x}$} the mean and \textit{$\sigma$} the standard deviation.
\lstinputlisting[linerange=74-79]{../BasicStatistics.py}

\subsection{Testing the functions}
I implemented a script file with the name \textit{BasicStatistics$\_$Test} which tests all the function with two data sets. 

\begin{itemize}
	\item 1, 2, 3, 4, 5
	\item 1, 2, 3, 4, 5, 6
\end{itemize}
                                                                                                                                                                                                                                                              
\subsection{Loading the data}
The data which I should analyse is stored in the file \textit{gauss.csv}. The data is one dimensional and I read the lines in the file with the function \textit{reader()} of the imported csv package. The data was then stored in the array \textit{dataSet}. 

\lstinputlisting[linerange=14-18]{../Assignment_2.py}

\subsection{Plotting the results}
For plotting the results, I created a plot with the \textit{.figure()} function of \textit{matplotlib} with the width of 8 inch and the height of 6 inch. I set the subtitle of the plot to \textit{Data Distribution} and created a subplot. We need three different subplots in the plot. One histogram and two scatter plots. 
First, I created the histogram with the function  \textit{.subplot(2,1,1)}. This subplot consumes two columns and one row and starts at the first position. I added the needed labels and plotted the values with the \textit{.hist()} function. I used 20 bins and the density property for the histogram. Also four vertical lines with the mean, median and the standard deviation with plus and minus was added to the subplot of the histogram. 
The other two subplots contain the raw and the standardized data, where each of the plots had three horizontal lines with the mean and the standard deviation. 

\section{Results}
The charts in Figure 1 display the result of the basic statistic functions. The histogram at the top displays the data distribution of the normalized data with the mean, median and standard deviation, the scatter chart at the bottom left the raw data with the mean and standard deviation and the scatter chart at the bottom right the normalized data with mean and standard deviation. 

\begin{figure}[htbp] %or use htbp to place it inside the text blocks
  \centering
  \includegraphics[width=\columnwidth]{pics/result.png}
  \caption{Illustrates the raw data in a Histogram and the raw and standardized data in a scatter chart}
  \label{fig:fibonacciPlot}
\end{figure}


\section{Conclusion}
After some clarification of the tasks, implementing the statistic function was straightforward. The code needed for the implementation could keep clean and minimalistic. Only creating the plot functions needed some additional lines of code and the function \textit{.tight$\_$layout()} to improve the readability of the charts and labels. 

\section*{APPENDIX}

\textit{BasicStatistics} class: 
\lstinputlisting{../BasicStatistics.py}
\newpage

\textit{BasicStatistics$\_$Test} class: 
\lstinputlisting{../BasicStatistics_Test.py}
\newpage

Plotting of the values:
\lstinputlisting{../Assignment_2.py}




%----------------------------------------------------------------------------------------
%	BIBLIOGRAPHY
%----------------------------------------------------------------------------------------

\printbibliography[title={Bibliography}] % Print the bibliography, section title in curly brackets

%----------------------------------------------------------------------------------------

\end{document}
