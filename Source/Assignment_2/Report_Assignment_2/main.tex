%%%%%%%%%%%%%%%%%%%%%%%%%%%%%%%%%%%%%%%%%
% Chair of Cyber-Physical-Systems
% Univ.-Prof. Dr. Elmar Rueckert
% Montanuniversität Leoben, Austria
% Latest Update: Feb. 2022
%
% Disclaimer: The materials and source code are for personal use only. The material is intended for educational purposes only. Reproduction of the material for any purposes other than what is intended is prohibited. The content is to be used for educational and non-commercial purposes only and is not to be changed, altered, or used for any commercial endeavor without the express written permission of Professor Rueckert. 
% 
% Original Version by Frits Wenneker, 28/2/17,  License: CC BY-NC-SA 3.0 (http://creativecommons.org/licenses/by-nc-sa/3.0/)
%%%%%%%%%%%%%%%%%%%%%%%%%%%%%%%%%%%%%%%%%

%----------------------------------------------------------------------------------------
%	PACKAGES AND OTHER DOCUMENT CONFIGURATIONS
%----------------------------------------------------------------------------------------

\documentclass[10pt, a4paper, twocolumn]{article} % 10pt font size (11 and 12 also possible), A4 paper (letterpaper for US letter) and two column layout (remove for one column)

\input{structure.tex} % Specifies the document structure and loads requires packages

%----------------------------------------------------------------------------------------
%	ARTICLE INFORMATION
%----------------------------------------------------------------------------------------

\title{Assignment II: Latex and Python Basics} % The article title

\author{
	\coursetitle{Exercises in Machine Learning (190.013), SS2022}
	\authorstyle{Stefan Nehl\textsuperscript{1}} % Authors
	\newline\newline % Space before institutions
	\textsuperscript{1}\textit{stefan-christopher.nehl@stud.unileoben.ac.at, MNr: 00935188}, \institution{Montanuniversität Leoben, Austria}\\ % Institution 1
	\newline\submissiondate{\today} % Add a date here
}

% Example of a one line author/institution relationship
%\author{\newauthor{John Marston} \newinstitution{Universidad Nacional Autónoma de México, Mexico City, Mexico}}


%----------------------------------------------------------------------------------------

\begin{document}
\input{python_code.tex} % To print Python code

\maketitle % Print the title

\thispagestyle{firstpage} % Apply the page style for the first page (no headers and footers)

%----------------------------------------------------------------------------------------
%	ABSTRACT
%----------------------------------------------------------------------------------------

\lettrineabstract{In the second assignment, I had to create basic statistic functions for calculating the mean, median, variance and standard deviation. Furthermore, I had to read in a CSV file, use the created functions and plot the values.}

%----------------------------------------------------------------------------------------
%	REPORT CONTENTS
%----------------------------------------------------------------------------------------

\section{Introduction}

The statistic functions I implemented were calculating the mean, median, variance, and standard deviation. Furthermore, I added a function for normalizing and standardizing a data set.  

\section{Implementation}
For the implementation I created the class \textit{BasicStatistics} with all the statistic functions. First, I implemented the mean. 

\subsection{Mean}
The mean is the average of a collection of numbers. The formula for the calculation is the following
\citep{meanCFI}: 
\[
mean = \frac{x\textsubscript{1} + x\textsubscript{2} + \ldots + x\textsubscript{n}} {n}
\]
Where n is the number of elements in the collection and x\textsubscript{n} the element on the position n. The calculation was implemented in the \textit{getMean} function. 
The function \textit{checkDataSet} checks if the \textit{dataSet} is empty or \textit{None}. If this is the case \textit{None} is returned.
\lstinputlisting[linerange=14-21]{../BasicStatistics.py}

\subsection{Median}
The Median is the middle value of a collection of numbers. The steps of the implemented algorithm are
\citep{medianCFI}: 
\begin{itemize}
	\item Sort the collection of numbers 
	\item Calculate the mid index 
	\item if the length of the collection is uneven: take the element with the mid index
	\item if the length is even calculate the median: 
	\[
	median = \frac{x\textsubscript{mid} + x\textsubscript{mid-1}} {2}
	\]
	\item return the result
\end{itemize}


\lstinputlisting[linerange=24-39]{../BasicStatistics.py}

\subsection{Variance}
The variance is the expected variation between values in a collection of numbers. 
The formula for the calculation of the variance it the following
\citep{varianceAndStandardDeviationStack}: 
\[
Variance\:\sigma\textsuperscript{2} = \frac{\sum_{i=0}^{n-1}(x\textsubscript{i} - \overline{x})}{n-1}
\]
Where \textit{n} is the number of elements, \textit{x\textsubscript{i}} the element on the index \textit{i} and \textit{$\overline{x}$} the mean. This formula uses the Bessel's correction for smaller numbers. Therefore, instead of dividing the aggregated values with n, I divided them with n-1 \citep{varianceAndStandardDeviationStack}. 

\lstinputlisting[linerange=42-53]{../BasicStatistics.py}

\subsection{Standard Deviation}
The standard deviation is the amount of the variation of a collection of numbers. 
The formula for the calculation of the standard deviation is
\citep{varianceAndStandardDeviationStack}: 
\[
Standard\:Deviation\:\sigma = \sqrt{\sigma\textsuperscript{2}}
\]

\lstinputlisting[linerange=56-62]{../BasicStatistics.py}


\subsection{Normalize Data}
The normalization is a process to transform the values in a collection of numbers to values between zero and one. The idea is that every value contributes equally to the further analysis. The calculation happens with the formula \citep{normalizeDataStatology}: 
\[
x\textsubscript{norm} = \frac{x\textsubscript{i} - x\textsubscript{min}} {x\textsubscript{max} - x\textsubscript{min}}
\]
Where \textit{x\textsubscript{norm}}, is the normalized value, \textit{x\textsubscript{i}} the value which needs to be normalized, \textit{x\textsubscript{min}} the min values and \textit{x\textsubscript{max}} the maximum value of the collection. 
\lstinputlisting[linerange=81-89]{../BasicStatistics.py}
The functions \textit{findMinValue} and \textit{findMaxValue} return the min or max value of the given collection. 

\subsection{Standardize Data}
Standardization is a preprocessing step, to standardize the range of values of a collection of numbers. I used the following formula to standardize the data
\citep{standardizeDataBuildIn}. 
\[
z = \frac{x\textsubscript{i} - \overline{x}}{\sigma}
\]
Where \textit{z} is the standardize value, \textit{$x\textsubscript{i}$} the value in the collection on index i, \textit{$\overline{x}$} the mean and \textit{$\sigma$} the standard deviation.
                                                                                                                                                                                                                                                             
\subsection{Loading the data}
\subsection{Plotting the results}
For plotting the results, I created a plot with the \textit{.figure()} function of \textit{matplotlib}. First, I set the subtitle of this plot to \textit{Fibonacci}, the x-axis label to \textit{Number} and the y-axis label to \textit{Value}. I created an array with the \textit{range()} function from python which contains the values from 1 to 30 and is needed for the x-axis values. Next, is set y-axis values to the results of the Fibonacci calculation, added the values to the plot with the  \textit{.plot()} function and displayed the plot with \textit{.show()}.

\section{Results}
Figure 1 displays the result of the Fibonacci calculation. The calculated values grow very fast as the numbers for the calculation rise. For example, the value for the number 15 is 610, and for 30, 832040. Table 1 displays the first and last three values of the Fibonacci calculation.    

\begin{figure}[t] %or use htbp to place it inside the text blocks
  \centering
  \includegraphics[width=\columnwidth]{pics/fibonacci_1.png}
  \caption{Illustrates the calculated values for the Fibonacci sequence from 1 to 30.}
  \label{fig:fibonacciPlot}
\end{figure}

\begin{table}[h]
         \label{tab:fibonacciValuesTable}
	\caption{Fibonacci Values}
	\centering
	\begin{tabular}{c c}
		\toprule
		Number & Calculated Value \\
		\midrule
		1 & 1 \\
		2 & 1 \\
		3 & 2 \\
		... & ... \\
		28 & 317811 \\
		29 & 514229 \\
		30 & 832040 \\
		\bottomrule
	\end{tabular}
\end{table}


\section{Conclusion}
The setup for the assignment was straightforward, and no issues appeared in the process. Also, the implementation of the algorithm happened without any problem. However, I had some trouble with creating the report because of the wrong build options in \textit{Texmark}. Changing the build options to \textit{PDFLaTex} solved this issue. 

\newpage

\section*{APPENDIX}

Fibonacci class: 
\lstinputlisting{../BasicStatistics.py}
\newpage

Plotting of the values:
\lstinputlisting{../Assignment_2.py}




%----------------------------------------------------------------------------------------
%	BIBLIOGRAPHY
%----------------------------------------------------------------------------------------

\printbibliography[title={Bibliography}] % Print the bibliography, section title in curly brackets

%----------------------------------------------------------------------------------------

\end{document}
