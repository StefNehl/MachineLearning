%%%%%%%%%%%%%%%%%%%%%%%%%%%%%%%%%%%%%%%%%
% Chair of Cyber-Physical-Systems
% Univ.-Prof. Dr. Elmar Rueckert
% Montanuniversität Leoben, Austria
% Latest Update: Feb. 2022
%
% Disclaimer: The materials and source code are for personal use only. The material is intended for educational purposes only. Reproduction of the material for any purposes other than what is intended is prohibited. The content is to be used for educational and non-commercial purposes only and is not to be changed, altered, or used for any commercial endeavor without the express written permission of Professor Rueckert. 
% 
% Original Version by Frits Wenneker, 28/2/17,  License: CC BY-NC-SA 3.0 (http://creativecommons.org/licenses/by-nc-sa/3.0/)
%%%%%%%%%%%%%%%%%%%%%%%%%%%%%%%%%%%%%%%%%

%----------------------------------------------------------------------------------------
%	PACKAGES AND OTHER DOCUMENT CONFIGURATIONS
%----------------------------------------------------------------------------------------

\documentclass[10pt, a4paper, twocolumn]{article} % 10pt font size (11 and 12 also possible), A4 paper (letterpaper for US letter) and two column layout (remove for one column)

\input{structure.tex} % Specifies the document structure and loads requires packages

%----------------------------------------------------------------------------------------
%	ARTICLE INFORMATION
%----------------------------------------------------------------------------------------

\title{Assignment I: Latex and Python Basics} % The article title

\author{
	\coursetitle{Exercises in Machine Learning (190.013), SS2022}
	\authorstyle{Stefan Nehl\textsuperscript{1}} % Authors
	\newline\newline % Space before institutions
	\textsuperscript{1}\textit{stefan-christopher.nehl@stud.unileoben.ac.at, MNr: 00935188}, \institution{Montanuniversität Leoben, Austria}\\ % Institution 1
	\newline\submissiondate{\today} % Add a date here
}

% Example of a one line author/institution relationship
%\author{\newauthor{John Marston} \newinstitution{Universidad Nacional Autónoma de México, Mexico City, Mexico}}


%----------------------------------------------------------------------------------------

\begin{document}
\input{python_code.tex} % To print Python code

\maketitle % Print the title

\thispagestyle{firstpage} % Apply the page style for the first page (no headers and footers)

%----------------------------------------------------------------------------------------
%	ABSTRACT
%----------------------------------------------------------------------------------------

\lettrineabstract{In the second assignment, I had to create basic statistic functions for calculating the mean, median, variance and standard deviation. Furthermore, I had to read in a CSV file, use the created functions and plot the values.}

%----------------------------------------------------------------------------------------
%	REPORT CONTENTS
%----------------------------------------------------------------------------------------

\section{Introduction}

The statistic functions I implemented were calculating the mean, median, variance, and standard deviation. So, first, I implemented the mean. 

\[
 f(y) =
\begin{cases}
	0 & {n = 0}\\
  	1 & {n = 1}\\
  	f(n) = f(n-1) + f(n-2) & n > 1
\end{cases}
\]
\citep{canaan2011all}

\section{Methods}
For the results, I followed the following steps: 

\begin{itemize}
	\item Installed PyCharm
	\item Created a project with a virtual environment 
	\item Installed packages, \textit{matplotlib} and \textit{numpy}
	\item Developed the algorithm
	\item Plotted the results
\end{itemize}
\textit{PyCharm} is already installed on my pc, and the creation of a project worked without any issues.   Furthermore, installing the packages \textit{matplotlib} and \textit{numpy} with PIP also worked. Because of this, I will focus in this report on the development of the algorithm

\subsection{Development of the algorithm}
First, I created a class \textit{Fibonacci} to implement the needed calculations. I defined a function called \textit{get$\_$fibonacci$\_$numbers(self, n)}. The parameter \textit{self} is the object of the class and \textit{n} the range of the Fibonacci calculation. The next step was to initialize the array, \textit{fibonacci$\_$numbers} of size \textit{n} with zeros. The package \textit{Numpy} provides a function for this operation called \textit{.zeros(n)}, where \textit{n} defines the size of the array. Next, I handled the base case for \textit{n < 0}, \textit{n = 0} and \textit{n = 1} with \textit{if} and \textit{elif}. The calculation of the Fibonacci \textit{n > 1} happens in the \textit{for-loop}. I iterated over the complete \textit{fibonacci$\_$numbers} array with the variable \textit{x} and implemented different cases. If \textit{x = 0}, the calculated value is 1, if \textit{x = 1} the calculated value is 1. I need this to calculate the Fibonacci for the third value. Otherwise, the access to the array position \textit{x - 2} would fail. The last step in the \textit{for-loop} is to handle the Fibonacci calculation itself. This calculation happens by accessing the values \textit{x - 1} and \textit{x - 2} in the \textit{fibonacci$\_$numbers} array and summing them up. The newly calculated value is then set to the position \textit{x} in the \textit{fibonacci$\_$numbers} array. The \textit{print} function in line number 26 only displays the values in the console. This function is not part of this assignment. After the loop, I return the \textit{fibonacci$\_$numbers} array.  

\subsection{Plotting the results}
For plotting the results, I created a plot with the \textit{.figure()} function of \textit{matplotlib}. First, I set the subtitle of this plot to \textit{Fibonacci}, the x-axis label to \textit{Number} and the y-axis label to \textit{Value}. I created an array with the \textit{range()} function from python which contains the values from 1 to 30 and is needed for the x-axis values. Next, is set y-axis values to the results of the Fibonacci calculation, added the values to the plot with the  \textit{.plot()} function and displayed the plot with \textit{.show()}.

\section{Results}
Figure 1 displays the result of the Fibonacci calculation. The calculated values grow very fast as the numbers for the calculation rise. For example, the value for the number 15 is 610, and for 30, 832040. Table 1 displays the first and last three values of the Fibonacci calculation.    

\begin{figure}[t] %or use htbp to place it inside the text blocks
  \centering
  \includegraphics[width=\columnwidth]{pics/fibonacci_1.png}
  \caption{Illustrates the calculated values for the Fibonacci sequence from 1 to 30.}
  \label{fig:fibonacciPlot}
\end{figure}

\begin{table}[h]
         \label{tab:fibonacciValuesTable}
	\caption{Fibonacci Values}
	\centering
	\begin{tabular}{c c}
		\toprule
		Number & Calculated Value \\
		\midrule
		1 & 1 \\
		2 & 1 \\
		3 & 2 \\
		... & ... \\
		28 & 317811 \\
		29 & 514229 \\
		30 & 832040 \\
		\bottomrule
	\end{tabular}
\end{table}


\section{Conclusion}
The setup for the assignment was straightforward, and no issues appeared in the process. Also, the implementation of the algorithm happened without any problem. However, I had some trouble with creating the report because of the wrong build options in \textit{Texmark}. Changing the build options to \textit{PDFLaTex} solved this issue. 

\newpage

\section*{APPENDIX}

Fibonacci class: 
\lstinputlisting{../BasicStatistics.py}
\newpage

Plotting of the values:
\lstinputlisting{../Assignment_2.py}




%----------------------------------------------------------------------------------------
%	BIBLIOGRAPHY
%----------------------------------------------------------------------------------------

\printbibliography[title={Bibliography}] % Print the bibliography, section title in curly brackets

%----------------------------------------------------------------------------------------

\end{document}
